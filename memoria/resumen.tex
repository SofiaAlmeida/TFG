\chapter*{Resumen}

En este trabajo se estudiarán algunos conceptos primordiales en el ámbito de la teoría de la información: entropía e información mutua, además de sus propiedades y conceptos relacionados. El objetivo principal de este proyecto es el estudio de estimadores, tanto de la entropía como de la información mutua, desde un punto de vista teórico, donde nos centraremos en los estimadores basados en los $k$ vecinos más cercanos y en los del tipo de Kozachenko - Leonenko, y práctico, donde realizaremos un análisis experimental de implementaciones de algunos de los estimadores anteriores desde diferentes perspectivas: análisis del error, estudio de las diferencias medias mediante \textit{T-test} y análisis de rendimiento. Las conclusiones obtenidas en el análisis comparativo de los diferentes estimadores a nivel práctico dependerán de la dimensión y del tamaño de las muestras.\\

\paragraph{Palabras clave:} entropía, información mutua, estimadores basados en los $k$ vecinos, estimadores de tipo de Kozachenko - Leonenko, Python.
